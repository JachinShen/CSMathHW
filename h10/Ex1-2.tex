
\begin{exercise}[From the video lecture]
   Recall the definition of the value of a flow: ${\rm val}(f) = \sum_{v \in V} f(s,v)$.
   Let $S \subseteq V$ be a set of vertices that contains $s$ but not $t$. Show that
   \begin{align*}
         {\rm val}(f) = \sum_{u \in S, v \in V \setminus S} f(u,v) \ .
   \end{align*}
   That is, the total amount of flow leaving $s$ equals the total amount of flow 
   going from $S$ to $V \setminus S$.
   \textbf{Remark.} It sounds obvious. However, find a formal proof that works with the 
   axiomatic definition of flows. 
\end{exercise}


\begin{exercise}
Let $G = (V,E,c)$ be a flow network.
  Prove that flow is ``transitive'' in the following sense: 
  If there is a flow from $s$ to $r$ of value $k$,
  and a flow from $r$ to $t$ of value $k$, then
  there is a flow from $s$ to $t$ of value $k$.
  \textbf{Hint.} The solution is extremely short. If you are trying
  something that needs more than 3 lines to write, you are on the wrong
  track.
\end{exercise}


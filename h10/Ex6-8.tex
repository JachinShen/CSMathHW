
\begin{exercise}
   Let $(G,s,t,c)$ be a flow network and $V_0, V_1, \dots, V_k$ be an optimal layering
   (that is, $k = {\rm dist}_G(s,t)$.
   Let $p$ be a path from $s$ to $t$ of length $k$. 
   Suppose we route some flow $f$ along $p$ (of some
   value $c_{\rm min} > 0$) and let $(G_f,s,t,c_f)$ be the residual network. Show that
   $V_0, V_1,\dots, V_k$ is a layering of $(G_f,s,t,c_f)$, too. Obviously, condition (1) and (2) in
   the definition of $k$-layerings still hold, so you only have to check  condition (3).
\end{exercise}

\textbf{Solution}

For any edge $(u,v)$ in path $p$ in $G$, $u \in V_i$, $v \in V_j$, $j \leq i + 1$.

Since the length of the path is $k$, $j = i + 1$

In the residual graph $G_f$, $(u,v)$ split to $(u,v)$ and $(v,u)$.

Obviously, $(u,v)$ still satisfies the condition (3).

For $(v,u)$, $i = j-1 \leq j+1$, thus it satisfies the condition (3), too.

Therefore, $V_0, V_1, \dots, V_k$ is a layering of $(G_f, s, t, c_f)$, too.


\begin{exercise}
   Show that every network $(G,s,t,c)$ has an optimal layering, provided there is a path
   from $s$ to $t$.
\end{exercise}

\textbf{Solution}

\textbf{Case 1}: When the provided path from $s$ to $t$ is the shortest-path with length $k$, according to \textbf{Exercies 10.5}, there is a $k$-layering. Thus, the network has an optimal layering.

\textbf{Case 2}: When the provided path from $s$ to $t$ is not the shortest-path, then there is another shortest-path from $s$ to $t$ with length $k$. According to \textbf{Exercies 10.5}, there is a $k$-layering. Thus, the network has an optimal layering.


\begin{exercise}
   Imagine we are in some iteration of the while-loop of the Ford-Fulkerson method.
   Let $V_0, \dots, V_k$ be an optimal layering of $(G,s,t,c)$. Show that after at most $m$
   iterations of the while-loop, $V_0,\dots,V_k$ ceases
   to be an optimal layering. \textbf{Remark.} Note that it is the {\em network} that changes from
   iteration to iteration of the while-loop, not the partition $V_0,\dots,V_k$. We consider
   the partition $V_0,\dots,V_k$ to be fixed in this exercise.
\end{exercise}

\textbf{Solution}

After every iteration, an edge is completely reversed. There are total $m$ edges, so after at most $m$ iterations, one edge of the shortest path must be reversed. Then there may be no path from $s$ to $t$ or a longer path. So $V_0, V_1, \dots, V_n$ ceases to be an optimal layering.

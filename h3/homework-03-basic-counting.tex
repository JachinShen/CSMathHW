% !TEX root = homework-03-basic-counting.tex
\documentclass[12pt,a4]{article}







\usepackage{graphicx,amsmath,amssymb,amsthm, boxedminipage,xcolor}

%\usepackage[lined,boxed]{algorithm2e}

\usepackage{algorithm}
\usepackage{algpseudocode}

%\usepackage{algorithmic}
\usepackage{algpseudocode}
\usepackage{amsmath}
\usepackage{graphics}
\usepackage{epsfig}

\newtheorem{theorem}{Theorem}[section]
\newtheorem{proposition}[theorem]{Proposition}
\newtheorem{lemma}[theorem]{Lemma}
\newtheorem{corollary}[theorem]{Corollary}
\newtheorem{definition}[theorem]{Definition}

\newtheorem*{theorem*}{Theorem}
\newtheorem*{lemma*}{Lemma}
\newtheorem*{proposition*}{Proposition}


\newtheorem{exercise}[theorem]{Exercise}
\newtheorem{exerciseD}[theorem]{*Exercise}
\newtheorem{exerciseDD}[theorem]{**Exercise}

\let\oldexercise\exercise
\renewcommand{\exercise}{\oldexercise\normalfont}

%\let\oldexerciseD\exerciseD
%\renewcommand{\exerciseD}{\oldexerciseD\normalfont}

%\let\oldexerciseDD\exerciseDD
%\renewcommand{\exerciseDD}{\oldexerciseDD\normalfont}

\newcommand{\E}{\mathbb{E}}
%\newcommand{\nth}[1]{#1^{\textsuperscript{th}}}
\newcommand{\scalar}[2]{\ensuremath{\langle #1, #2\rangle}}
\newcommand{\floor}[1]{\left\lfloor #1 \right\rfloor}
\newcommand{\ceil}[1]{\left\lceil #1 \right\rceil}
\newcommand{\norm}[1]{\|#1\|}
\newcommand{\pfrac}[2]{\left(\frac{#1}{#2}\right)}
\newcommand{\nth}[1]{#1^{\textsuperscript{th}}}
\newcommand{\core}{\textnormal{core}}



\newif\ifsolution

\solutionfalse

\newcommand{\answer}[1]{
\ifsolution
{\color{blue} #1}
\else
\fi
}



\newcommand{\poly}{\textnormal{poly}}
\newcommand{\quasipol}{\textnormal{quasipol}}
\newcommand{\ssubexp}{\textnormal{stronglySubExp}}
\newcommand{\wsubexp}{\textnormal{weaklySubExp}}
\newcommand{\simplyexp}{\textnormal{E}}
\newcommand{\expo}{\textnormal{Exp}}



\newcommand{\N}{\mathbb{N}}
\newcommand{\nn}{\mathbb{N}_0^n}
\newcommand{\R}{\mathbb{R}}
\newcommand{\Z}{\mathbb{Z}}


\definecolor{darkgreen}{rgb}{0,0.6,0}


\date{}

\title{
  Mathematical Foundations \\of \\Computer Science\\
  \vspace{3mm}
{\normalsize CS 499,	Shanghai Jiaotong University,  Dominik Scheder}
}

\begin{document}

\maketitle

%\begin{quotation}
%  You are welcome to discuss the exercises in the discussion
%  forum. Please take them serious. Doing the exercises is as important
%  than watching the videos.
%
%  I intentionally included very challenging exercises and marked them
%  with one or two ``$*$''. No star means you should be able to solve
%  the exercises without big problems once you have understood
%  the material from the video lecture. One star means it requires 
%  significant additional thinking. Two stars means it is not 
%  unlikely that you will fail to solve them, even once you have understood
%  the material and thought a lot about the exercise. Don't feel bad
%  if you fail. Failure is part of learning.
%
%  This is the first time this course is online. Thus there might be mistakes
%  (typos or more serious conceptual mistakes) in the exercises. I will be 
%  grateful if you point them out to me!
%\end{quotation}




\setcounter{section}{2}


\begin{itemize}
 \item Homework assignment published on Tuesday, 2018-03-13
 \item Submit questions and first solutions by Sunday, 2018-03-18, 12:00 by email to dominik.scheder@gmail.com  and the TAs.
 \item You will receive feedback by Wednesday, 2018-03-21
 \item Revise your solution and submit your final solution by Sunday, 2018-03-25 by  email to dominik.scheder@gmail.com and the TAs.
\end{itemize}



\section{Basic Counting}

A function $[m] \rightarrow [n]$ is {\em monotone} if $f(1) \leq f(2) \leq \dots \leq f(m)$.
It is  {\em strictly monotone} if $f(1) < f(2) < \dots < f(m)$.

\begin{exercise}
   Find and justify a closed formula for the number of strictly 
   monotone functions from $[m]$ to $[n]$. 
\end{exercise}


\begin{exercise}
   Find and justify a closed formula for the number of monotone functions from $[m]$ to $[n]$. 
\end{exercise}

\textbf{Remark.} By ``closed'' I mean something using expressions like $\times$, $+$, 
${n \choose k}$, $n!$, but not $\sum$ or $\prod$. 
For example, ${n \choose k^2}$ is a closed formula but
$\sum_{k=0}^n {n \choose k}$ is not.


\begin{exercise}
  Prove that $\sum_{k=0}^n {n \choose k}^2 = {2n \choose n}$ for every $n \geq 0$ by finding a combinatorial interpretation.
\end{exercise}  

\begin{exercise}[From the textbook]
   Find a closed formula for $\sum_{k=m}^n {k \choose m}{n \choose k}$ and prove it combinatorially, i.e., by giving an 
   interpretation.
\end{exercise}

\begin{exercise}
   Let $B_n$ be the number of partitions of the set $[n]$ (this is the same as the number
   of equivalence relations on $[n]$). This is called the Bell number, thus we 
   denote it $B_n$.
   Prove that the following recursive formula
   for $B_n$ is correct:
   \begin{align*}
     B_0 & = 1 \\
     B_{n+1} & = \sum_{k=0}^n {n \choose k} B_k  \ .
   \end{align*}
\end{exercise}

\begin{exercise}
  Let $P_n$ be the number of ways to write the natural number $n$ as a sum $
  a_1 + a_2 + \cdots + a_k$ such that
  $1 \leq a_1 \leq a_2 \leq \dots \leq a_k$. For example, $3$ can be written
  as $3$, $2 + 1$, and $1 + 1 + 1$, so $P_3 = 3$. 
   Find a recursive formula for $P_n$.\\
   
   \textbf{Remark.} The formula might not be as simple as the above for $B_n$. Be creative!  
   Start by writing a simple recursive program that computes $P_n$.
\end{exercise}



\end{document}


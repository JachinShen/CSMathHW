%% ================================================================================
%% This LaTeX file was created by AbiWord.                                         
%% AbiWord is a free, Open Source word processor.                                  
%% More information about AbiWord is available at http://www.abisource.com/        
%% ================================================================================

\documentclass[a4paper,portrait,12pt]{article}
\usepackage[latin1]{inputenc}
\usepackage{calc}
\usepackage{setspace}
\usepackage{fixltx2e}
\usepackage{graphicx}
\usepackage{multicol}
\usepackage[normalem]{ulem}
%% Please revise the following command, if your babel
%% package does not support zh-CN
\usepackage[zh]{babel}
\usepackage{color}
\usepackage{hyperref}
 
\begin{document}


\begin{flushleft}
Mathematical Foundations
\end{flushleft}


\begin{flushleft}
of
\end{flushleft}


\begin{flushleft}
Computer Science
\end{flushleft}


\begin{flushleft}
CS 499, Shanghai Jiaotong University, Dominik Scheder
\end{flushleft}





6





\begin{flushleft}
Graph Theory Basics
\end{flushleft}


\begin{flushleft}
$\bullet$ Homework assignment published on Monday, 2018-04-02.
\end{flushleft}


\begin{flushleft}
$\bullet$ Submit first solutions and questions by Sunday, 2018-04-08, 12:00, by
\end{flushleft}


\begin{flushleft}
email to dominik.scheder@gmail.com and to the TAs.
\end{flushleft}


\begin{flushleft}
$\bullet$ You will receive feedback by Wednesday, 2018-04-11.
\end{flushleft}


\begin{flushleft}
$\bullet$ Submit final solution by Sunday, 2018-04-15 to me and the TAs.
\end{flushleft}





\begin{flushleft}
Let G = (V, E) and H = (V 0 , E 0 ) be two graphs. A graph isomorphism
\end{flushleft}


\begin{flushleft}
from G to H is a bijective function f : V $\rightarrow$ V 0 such that for all u, v $\in$ V
\end{flushleft}


\begin{flushleft}
it holds that \{u, v\} $\in$ E if and only if \{f (u), f (v)\} $\in$ E 0 . If such a function
\end{flushleft}


\begin{flushleft}
exists, we write G $\sim$
\end{flushleft}


\begin{flushleft}
= H and say that G and H are isomorphic. In other
\end{flushleft}


\begin{flushleft}
words, G and H being isomorphic means that they are identical up to the
\end{flushleft}


\begin{flushleft}
names of its vertices.
\end{flushleft}


\begin{flushleft}
Obviously, every graph G is isomorphic to itself, because the identity
\end{flushleft}


\begin{flushleft}
function f (u) = u is an isomorphism. However, there might be several
\end{flushleft}


\begin{flushleft}
isomorphisms f from G to G itself. We call such an isomorphism from G to
\end{flushleft}


\begin{flushleft}
itself an automorphism of G.
\end{flushleft}


\begin{flushleft}
Exercise 6.1. For each of the graphs below, compute the number of automorphisms it has.
\end{flushleft}


1





\begin{flushleft}
\newpage
Justify your answer!
\end{flushleft}


\begin{flushleft}
Consider the n-dimensional Hamming cube Hn . This is the graph with
\end{flushleft}


\begin{flushleft}
vertex set \{0, 1\}n , and two vertices x, y $\in$ \{0, 1\}n are connected by an edge
\end{flushleft}


\begin{flushleft}
if they differ in exactly one edge. For example, the right-most graph in the
\end{flushleft}


\begin{flushleft}
figure above is H3 .
\end{flushleft}


\begin{flushleft}
Exercise 6.2. Show that Hn has exactly 2n · n! automorphisms. Be careful:
\end{flushleft}


\begin{flushleft}
it is easy to construct 2n · n! different automorphisms. It is more difficult to
\end{flushleft}


\begin{flushleft}
show that there are no automorphisms other than those.
\end{flushleft}


\begin{flushleft}
A graph G is called asymmetric if the identity function f (u) = u is the
\end{flushleft}


\begin{flushleft}
only automorphism of G. That is, if G has exactly one automorphism.
\end{flushleft}


\begin{flushleft}
Exercise 6.3. Give an example of an asymmetric graph on six vertices.
\end{flushleft}


\begin{flushleft}
Exercise 6.4. Find an asymmetric tree.
\end{flushleft}


\begin{flushleft}
For a graph G = (V, E), let Ḡ := V,
\end{flushleft}


\begin{flushleft}
graph.
\end{flushleft}





\begin{flushleft}
V
\end{flushleft}


2





?





?


\begin{flushleft}
\ensuremath{\backslash} E denote its complement
\end{flushleft}





\begin{flushleft}
Its complement H̄.
\end{flushleft}





\begin{flushleft}
A graph H on six vertices
\end{flushleft}





\begin{flushleft}
We call a graph self-complementary if G $\sim$
\end{flushleft}


\begin{flushleft}
= Ḡ. The above graph is not selfcomplementary. Here is an example of a self-complementary graph:
\end{flushleft}





\begin{flushleft}
The pentagon G.
\end{flushleft}





\begin{flushleft}
Ḡ, the pentagram.
\end{flushleft}





2





\begin{flushleft}
\newpage
Exercise 6.5. Show that there is no self-complementary graph on 999 vertices.
\end{flushleft}


\begin{flushleft}
Exercise 6.6. Characterize the natural numbers n for which there is a selfcomplementary graph G on n vertices. That is, state and prove a theorem
\end{flushleft}


\begin{flushleft}
of the form {``}There is a self-complementary graph on n vertices if and only
\end{flushleft}


\begin{flushleft}
if n $<$put some simple criterion here$>$.''
\end{flushleft}





3





\newpage



\end{document}

\documentclass[12pt,a4]{article}







\usepackage{graphicx,amsmath,amssymb,amsthm, boxedminipage,xcolor}

%\usepackage[lined,boxed]{algorithm2e}

\usepackage{algorithm}
\usepackage{algpseudocode}

%\usepackage{algorithmic}
\usepackage{algpseudocode}
\usepackage{amsmath}
\usepackage{graphics}
\usepackage{epsfig}

\newtheorem{theorem}{Theorem}[section]
\newtheorem{proposition}[theorem]{Proposition}
\newtheorem{lemma}[theorem]{Lemma}
\newtheorem{corollary}[theorem]{Corollary}
\newtheorem{definition}[theorem]{Definition}

\newtheorem*{theorem*}{Theorem}
\newtheorem*{lemma*}{Lemma}
\newtheorem*{proposition*}{Proposition}


\newtheorem{exercise}[theorem]{Exercise}
\newtheorem{exerciseD}[theorem]{*Exercise}
\newtheorem{exerciseDD}[theorem]{**Exercise}

\let\oldexercise\exercise
\renewcommand{\exercise}{\oldexercise\normalfont}

%\let\oldexerciseD\exerciseD
%\renewcommand{\exerciseD}{\oldexerciseD\normalfont}

%\let\oldexerciseDD\exerciseDD
%\renewcommand{\exerciseDD}{\oldexerciseDD\normalfont}

\newcommand{\E}{\mathbb{E}}
%\newcommand{\nth}[1]{#1^{\textsuperscript{th}}}
\newcommand{\scalar}[2]{\ensuremath{\langle #1, #2\rangle}}
\newcommand{\floor}[1]{\left\lfloor #1 \right\rfloor}
\newcommand{\ceil}[1]{\left\lceil #1 \right\rceil}
\newcommand{\norm}[1]{\|#1\|}
\newcommand{\pfrac}[2]{\left(\frac{#1}{#2}\right)}
\newcommand{\nth}[1]{#1^{\textsuperscript{th}}}
\newcommand{\core}{\textnormal{core}}



\newif\ifsolution

\solutionfalse

\newcommand{\answer}[1]{
\ifsolution
{\color{blue} #1}
\else
\fi
}



\newcommand{\poly}{\textnormal{poly}}
\newcommand{\quasipol}{\textnormal{quasipol}}
\newcommand{\ssubexp}{\textnormal{stronglySubExp}}
\newcommand{\wsubexp}{\textnormal{weaklySubExp}}
\newcommand{\simplyexp}{\textnormal{E}}
\newcommand{\expo}{\textnormal{Exp}}



\newcommand{\N}{\mathbb{N}}
\newcommand{\nn}{\mathbb{N}_0^n}
\newcommand{\R}{\mathbb{R}}
\newcommand{\Z}{\mathbb{Z}}


\definecolor{darkgreen}{rgb}{0,0.6,0}


\date{}

\title{
  Mathematical Foundations \\of \\Computer Science\\
  \vspace{3mm}
{\normalsize CS 499,	Shanghai Jiaotong University,  Dominik Scheder}
}

\begin{document}

\maketitle

%\begin{quotation}
%  You are welcome to discuss the exercises in the discussion
%  forum. Please take them serious. Doing the exercises is as important
%  than watching the videos.
%
%  I intentionally included very challenging exercises and marked them
%  with one or two ``$*$''. No star means you should be able to solve
%  the exercises without big problems once you have understood
%  the material from the video lecture. One star means it requires 
%  significant additional thinking. Two stars means it is not 
%  unlikely that you will fail to solve them, even once you have understood
%  the material and thought a lot about the exercise. Don't feel bad
%  if you fail. Failure is part of learning.
%
%  This is the first time this course is online. Thus there might be mistakes
%  (typos or more serious conceptual mistakes) in the exercises. I will be 
%  grateful if you point them out to me!
%\end{quotation}




\setcounter{section}{1}


\begin{itemize}
 \item Homework assignment published on Monday, 2018-03-05.
 \item Work on it and submit a first solution or questions by Sunday, 2018-03-11, 12:00 by
 email to me and the TAs.
 \item You will receive feedback by Wednesday, 2018-03-14.
 \item Submit your final solution by Sunday, 2018-03-18 to me and the TAs.
\end{itemize}



\section{Partial Orderings}



\subsection{Equivalence Relations as a Partial Ordering}

An equivalence relation $R \subseteq V \times V$ is basically the same as a partition
of $V$. A {\em partition} of $V$ is a set $\{V_1,\dots,V_k\}$ where
(1) $V_1 \cup \dots \cup V_k = V$ and (2) the $V_i$ are pairwise disjoint,
i.e., $V_i \cap V_j  = \emptyset$ for $1 \leq i < j \leq k$. For example,
$\{ \{1\}, \{2,3\}, \{4\} \}$ is a partition of $\{1,2,3,4\}$ but
$\{ \{1\}, \{2,3\}, \{1,4\}\}$ is not.



\begin{exercise}
  Let $E_4$ be the set of all equivalence relations on $\{1,2,3,4\}$. Note that
  $E_4$ is ordered by set inclusion, i.e.,
  \begin{align*}
     (E_4, \{ (R_1,R_2) \in E_4 \times E_4 \ | \ R_1 \subseteq R_2 \} )
  \end{align*}
  is a partial ordering.
  \begin{enumerate}
    \item Draw the Hasse diagram of this partial ordering in a  nice way.
    \item What is the size of the largest chain?
    \item What is the size of the largest antichain?
  \end{enumerate}
  \end{exercise}


\subsection{Chains and Antichains}

Define the partially ordered set $(\nn, \leq)$ as follows:
$x \leq y$ if $x_i \leq y_i$ for all $1 \leq i \leq n$. For example,
$(2,5,4) \leq (2,6,6)$ but $(2,5,4) \not \leq (3,1,1)$.

\begin{exercise}
  Consider the infinite partially ordered set $(\nn, \leq)$.
  \begin{enumerate}
  \item
    Which elements are minimal? Which are maximal?
    \par The minimal element is $(0,0,0,\cdots,0)$.(There are $n 0$s in the element.)
    \par No element is maximal.
  \item Is there a minimum? A maximum?
    \par The minimum element is $(0,0,0,\cdots,0)$.(There are $n 0$s in the element.)
    \par No element is maximum.
  \item Does it have an infinite chain?
    \par Yes.
    \par There is an example: $\{(0,0,0,\cdots,0),(1,0,0,\cdots,0),(1,1,0,\cdots,0),\cdots,(1,1,1,\cdots,1)\}$
  \item Does it have arbitrarily large antichains? That is, can you find an
  antichain $A$ of size $|A| = k$ for every $k \in \N$?
    \par Yes. We consider an antichain like this:
    \begin{center}
    $\{(1,0,0,\cdots,0),(0,1,0,\cdots,0),
    (0,0,1,\cdots,0),\cdots,(0,0,0,\cdots,1)\}$
    \end{center}
    For the $k^{th}$ element, there is only one $1$ in the $k^{th}$ position, and other positions are all occupied by $0$. And the antichain consists of these $k$ elements.
\end{enumerate}
\end{exercise}

\begin{exerciseD}
  Does every infinite subset $S \subseteq \nn$ contain
  an infinite chain?
\end{exerciseD}

\begin{proof}
\par
Base case $n=1$: Apparently, every two elements in set $N_0^0$ is comparable since there is only one dimension. If there exists an infinite subset $S \subseteq \N_0^0$, the subset $S$ itself is an infinite chain.
So, the theorem holds when $n=1$. \\
Inductive hypothesis: \\
\indent Suppose the theorem holds for all values of $n$ up to some $k$, $k \geq 1$.\\
Inductive step: \\
\indent Let $n=k+1$. If there exists an infinite subset $S \subseteq \N_0^{k+1}$, note
\begin{equation}
\begin{aligned}
S_1 &= \{ (a_1, a_2, ..., a_k) &| (a_1, a_2, ..., a_k, a_{k+1}) \in S \}\\
S_2 &= \{ a_{k+1} &| (a_1, a_2, ..., a_k, a_{k+1}) \in S \}
\end{aligned}
\end{equation}
\indent Since $S$ is infinite, at least one of $S_1, S_2$ is infinite. \\
\indent Suppose $S_1$ is infinite, according to inductive hypothesis, there is an infinite chain $C_k$ for $S_1 \subseteq \N_0^k$.
\begin{equation}
\begin{aligned}
C_k &= (A_1, A_2, \cdots), A_1 \leq A_2 \leq \cdots \\
A_i &= (a_{i1}, a_{i2}, \cdots, a_{ik})
\end{aligned}
\end{equation}
\indent Now we construct an infinite chain $C_{k+1}$ for $S \subseteq \N_0^{k+1}$.  Take $b \in S_2$, we append every $A_i$ with $b$ to get $B_i$.
\begin{equation}
\begin{aligned}
B_i &= (a_{i1}, a_{i2}, \cdots, a_{ik}, b)\\
C_{k+1} &= (B_1, B_2, \cdots), B_1 \leq B_2 \leq \cdots
\end{aligned}
\end{equation}
\indent So $C_{k+1}$ is an infinite chain for $S \subseteq \N_0^{k+1}$.\\
\indent Now suppose $S_1$ is finite and $S_2$ is infinite. Notice that $S_2$ itself is an infinite chain. We take $(a_1, a_2, ..., a_k) \in S_1$ and we can construct an infinite chain for $S \subseteq \N_0^{k+1}$ in a similar way.

So, the theorem holds for $n=k+1$.
By the principle of mathematical induction, the theorem holds for all $n \in \mathbb{N}$.
\end{proof}

\begin{exercise}
  Show that $(\nn,\leq)$ has no infinite antichain. \textbf{Hint.} Use
  the previous exercise.
\end{exercise}
\begin{proof}
We proof it by contradiction. Suppose there is an infinite antichain which is also a subset of $\nn$. But from Exercise 2.3, it is clear to us that every infinite subset $S \subseteq \nn$ contain an infinite chain. So there is a contradiction. Consequently, $(\nn,\leq)$ has no infinite antichain.
\end{proof}



Consider the induced ordering on $\{0,1\}^n$. That is, for $x,y\in \{0,1\}^n$
we have $x \leq y$ if $x_i \leq y_i$ for every coordinate $i \in [n]$.

\begin{exercise}
 Draw the Hasse diagrams of $(\{0,1\}^n, \leq)$ for $n=2,3$.
\end{exercise}

\begin{exercise}
  Determine the maximum, minimum, maximal, and minimal elements of
  $\{0,1\}^n$.
\end{exercise}
\begin{center}
Maximum element:$(1,1,1,\cdots,1)$
\par Maximal element:$(1,1,1,\cdots,1)$
\par Minimum element:$(0,0,0,\cdots,0)$
\par Minimal element:$(0,0,0,\cdots,0)$
\end{center}
\begin{exercise}
  What is the longest chain of $\{0,1\}^n$?
  \par One of the examples is as follows:
  \begin{center}
  $\{(0,0,0,\cdots,0),(1,0,0,\cdots,0),
  (1,1,0,\cdots,0),\cdots,(1,1,1,\cdots,1)\}$
  \end{center}
\end{exercise}


\begin{exerciseDD}
  What is the largest antichain of $\{0,1\}^n$?
\end{exerciseDD}



\subsection{Infinite Sets}

In the lecture (and the lecture notes) we have showed that $\N \times \N \cong \N$, i.e.,
there is a bijection $f: \N \times \N \rightarrow \N$. From this, and by induction, it follows
quite easily that $\N^k \cong \N$ for every $k$.

\begin{exercise}
   Consider $\N^*$, the set of all finite sequences of natural numbers, that is,
   $\N^* = \{\epsilon\} \cup \N \cup \N^2 \cup \N^3 \cup \dots$. Here,
   $\epsilon$ is the empty sequence. Show that $\N \cong \N^*$ by defining
   a bijection $\N \rightarrow \N^*$.
\end{exercise}

\begin{exercise}
   Show that $R \cong R \times R$. \textbf{Hint:} Use the fact that
   $R \cong \{0,1\}^{\N}$ and thus show that $\{0,1\}^{\N} \cong \{0,1\}^{\N} \times \{0,1\}^{\N}$.
\end{exercise}

\begin{exercise}
  Consider $\R^{\N}$, the set of all infinite sequences $(r_1, r_2, r_3,\dots)$ of real numbers.
  Show that $\R \cong \R^{\N}$. \textbf{Hint:} Again, use the fact that $\R \cong \{0,1\}^{\N}$.
\end{exercise}

Next, let us view $\{0,1\}^{\N}$ as a partial ordering: given two elements $\mathbf{a}, \mathbf{b} \in \{0,1\}^{\N}$,
that is, sequences $\mathbf{a} = (a_1,a_2,\dots)$ and $\mathbf{b} = (b_1,b_2,\dots)$, we define
$\mathbf{a} \leq \mathbf{b}$ if $a_i \leq b_i$ for all $i \in \N$. Clearly,
$(0,0,\dots)$ is the minimum element in this ordering and $(1,1,\dots)$ the maximum.\\

\begin{exercise}
   Give a countably infinite chain in $\{0,1\}^{\N}$. Remember that a set $A$ is countably infinite
   if $A \cong \N$.
\end{exercise}

\begin{exercise}
   Find a countably infinite antichain in $\{0,1\}^{\N}$.
\end{exercise}

\begin{exercise}
   Find an uncountable antichain in $\{0,1\}^{\N}$. That is, an antichain $A$ with $A \cong \R$.
\end{exercise}

\begin{exerciseDD}
   Find an uncountable chain in $\{0,1\}^{\N}$. That is, an antichain $A$ with $A \cong \R$.
\end{exerciseDD}





\end{document}




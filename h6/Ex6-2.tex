%\begin{exercise}


%ex 6-2


%\end{exercise}

\begin{proof}

First we choose a vertex in $G$ to be corresponding to our first vertex, say vertex $1$. There is $2^n$ ways. Note that $n$ adjacent vertexes to it can uniquely form a hyperplane and they are all symmetrical. So we arrange them to the adjacent vertexes and there is only $n!$ ways since all the hyperplanes are unique.
As a result, the number of automorphism is $2^nn!$.

 \par We prove there are no automorphisms other than those by contradiction. We assume that there is more than $2^n\times n$ ways automorphisms. So for the first vertex, there is more than $n!$ ways for the next adjacent vertices to be put. So there must be a repeat for it. But as we state above, for each way to put the adjacent vertices, the next level of adjacent vertices are unique. So there is a contradiction. 
 \par there is an example. For vertex$(1,1,1,1,1)$, the adjacent vertices are
 \begin{center}
 $(1,1,1,1,0),(1,1,1,0,1),(1,1,0,1,1),(1,0,1,1,1),(0,1,1,1,1)$,
  \end{center}
  with $5!$ ways to put them. We can know that the only vertice $(1,1,1,1,0),(1,1,1,0,1)$ are connected to is$(1,1,1,0,0)$, so if there is a repeat for the way the $n$ adjacent vertices to put, it must be the same, which is a contradiction.
\end{proof}
